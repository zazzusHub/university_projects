\documentclass[serif]{beamer}

\usepackage[T1]{fontenc}
\usepackage[sfdefault]{FiraSans}
\usepackage[english]{babel}
\usepackage[utf8]{inputenc}
\usepackage[separate-uncertainty=true]{siunitx}
\usepackage{amsmath}
\usepackage{amssymb}
\usepackage{amsfonts}
\usepackage{mathrsfs}
\usepackage{mathtools}
\usepackage{commath}
\usepackage{braket}
\usepackage{graphics}
\usepackage[euler-digits]{eulervm}
\usepackage{pifont} 
\usepackage{comment}

\usepackage[backend=biber]{biblatex}

\renewcommand*{\bibfont}{\footnotesize}
\addbibresource{biblio.bib}

\usepackage{booktabs}

\usetheme{Madrid}
\usepackage{xcolor}
\definecolor{OpenMCRed}{RGB}{163,31,52}
\definecolor{OpenMCGray}{RGB}{140,140,140}

\setbeamercolor{normal text}{fg=black}
%\setbeamercolor{frametitle}{fg=OpenMCRed}
\setbeamercolor{title}{fg=black, bg=OpenMCGray}
%\setbeamercolor{item}{fg=OpenMCDarkGray}
\setbeamercolor{structure}{fg=OpenMCRed}

\setbeamercolor{frametitle}{fg=black, bg=OpenMCGray}

\setbeamertemplate{frametitle}[default][center]
\title[ATF's Neutronic]{
   Accident Tolerant Fuels: Neutronic Aspects
}
\titlegraphic{
    \begin{columns}
        \begin{column}{0.5\textwidth}
            \centering
             \includegraphics[scale=0.17]{img/cherub.png}
        \end{column}
        \begin{column}{0.5\textwidth}
            \centering
             \includegraphics[scale=0.15]{openmc.png}
        \end{column} 
    \end{columns}
        
}
\author[F. Zazzu]{
    Francesco Zazzu \\
}
\date[17 febbraio 2025]{
    A.Y. 2024/2025
}
\institute[]{
    MSc in Nuclear Engineering \\
    Dipartimento di Ingegneria Civile e Industriale  \\
    Università di Pisa
}

\definecolor{OpenMCRed}{RGB}{163,31,52}
%\setbeamercolor{titlelike}{bg=unipibl}
\setbeamerfont{title}{series=\bfseries}

\useoutertheme[subsection=false]{miniframes}

\begin{document}

\frame{\titlepage}
\logo{
             \includegraphics[height=0.6cm]{openmc.png}
        }
\section{Introduction}
\begin{frame}{Introduction}
   

As a result of the Fukushima Daiichi accident, enhancing the safety of fuel and cladding became a topic of serious discussion.
    
\begin{itemize}
    \item[\ding{227}] Why are Accident Tolerant Fuels being developed, and \textbf{how do they compare with conventional fuels}?
    \item[\ding{227}] What are the \textbf{neutronic characteristics} of ATFs?
    \item[\ding{227}] How can \textbf{ Monte Carlo simulations} support the development of ATFs?
\end{itemize}

The aim of this work is to answer these questions through the use of
\textbf{OpenMC}, a community-developed Monte Carlo neutron and photon \textbf{transport} simulation code.
\end{frame}
%\begin{frame}{Accident Tolerant Fuel Objectives and Constrains} The \textbf{primary aim} 
%    of accident tolerant fuel is to provide, respect the current fuel system design:
%\begin{itemize}
%    \item \textbf{lower hydrogen generation rate};
%    \item \textbf{reduction in the initial and residual stored energy} in the core.
%\end{itemize}
%
%ATFs candidates need to achieve their objectives following some \textbf{constrains}:
%\begin{itemize}
%    \item \textbf{Maintaining or improving the fuel performance}, from the neutronic point of view implies flat power distribution, lower peaking factors, comparable feedback coefficient;
%    \item  \textbf{Backward Compatibility}, compatible with existing LWR designs.
%\end{itemize}
%Main character of this presentation: \textbf{FeCrAl} alloys.
%\end{frame}

\begin{frame}{FeCrAl alloy vs Zircaloy}
\only<1>{
Relevant properties and characteristics:
    \begin{itemize}
        \item \textbf{compatibility with the coolant chemistry}; 
        \item \textbf{lower oxidation rate} than Zircaloy;
        \item  \textbf{hydrogen has a higher mobility} in FeCrAl alloys and will not accumulate in the alloy, \textbf{reducing the embrittlement};
        \item \textbf{higher specific heat} than Zircaloy that \textbf{decreases the peak cladding temperature}.
    \end{itemize}
    Challenges:
    \begin{itemize}
        \item \textbf{permeability of tritium}, which poses a \textbf{radiation concern};
        \item increased \textbf{parasitic neutron absorption} of the FeCrAl.
    \end{itemize} 
}
\only<2>{
\tiny% Reduce font size for better fit
\vspace{-27pt}
\begin{table}
    \centering
    \renewcommand{\arraystretch}{1.2}
    \resizebox{\linewidth}{!}{  % Resize table to fit frame
    \begin{tabular}{lcc}
        \toprule
        \textbf{Property} & \textbf{Zircaloy-4
        } & \textbf{FeCrAl} \\
        \midrule
        Melting Point (°C) & 1850 & 1500 \\
        Specific Heat  (J/kg·K)  & 285-368 [25°C-700°C]& 480-710 [20°C-600°C]\\
        Thermal Conductivity (W/m·K) & 14.5-14.2 [25°-425°]& 11-21 [50°C-600°C
        ]\\
        Thermal Neutron Absorption Cross Section (barns) & 0.18-0.20 & 1.3-2.0 \\
        Density (g/cm³) & 6.52 & 7.1 \\
        \midrule
        \textbf{Component} & & \\
        Zr & 98.43\%& - \\
        Sn & 1.2\% & -\\
        Fe & 0.21\% & 78.68\%\\
        Cr &0.12\% & 13.02\%\\
        Al &-&5.1\%\\
        Mo &-&2.1\% \\
        Si &-&2\%\\
        Nb &-&1\%\\
        Y &-&0.3\%\\
        C &-&0.003\%\\
        \bottomrule

    \end{tabular}
    }
    \label{tab:cladding_comparison}
\end{table}
}
 \only<3>{ 
    \vspace{-22pt}
    \begin{columns} 
       \begin{column}{0.85\textwidth}
            \centering
            \includegraphics[width=\columnwidth]{img/fecral_abs_cs.png}
        \end{column}   
    \end{columns}
}
\end{frame}

\begin{frame}{On OpenMC}
\begin{itemize}
    \item After initializing the particle and determining its current cell,
            the \textbf{distance to the nearest boundary} $d_b$ of current cell is determined.
    \item The \textbf{distance to the next collision} is extrapolated thanks to a \textbf{pseudo-random} number $\xi$, 
        $d$ is then compared with $d_b$: 
    \[
    d=-\frac{\ln \xi}{\Sigma _t} ,~ \xi \in [0,1).
    \]

    \item To determine the \textbf{probability of interaction} of the particle \textbf{with each nuclide} $i$: 
    \[P(i) = \frac{\Sigma_{t,i}}{\Sigma_t}.\]
    
    \item A \textbf{reaction} for that nuclide is randomly sampled based on the microscopic cross sections: 
    \[P(x) = \frac{\sigma_x}{\sigma_t}.\]       
\end{itemize}
\end{frame}

\section{Initialization} 
\begin{frame}{Materials and Geometries}

    
\begin{columns}
    \begin{column}{0.31\textwidth}
        \vspace{-5pt}
        \centering
        \includegraphics[width=\columnwidth]{img/assembly_ref.png}
    \end{column}
    \begin{column}{0.6\textwidth}
        Some \textbf{approximations}:
        \begin{itemize}
            \item simplified geometries;
            \item only helium is present inside the plenum;
            \item the fuel pellet stack has been constructed as a single column. 
            \item burnable poisons are not dispersed in the water. 
        \end{itemize}
    \end{column}
\end{columns}

     
\end{frame}

\begin{frame}{Materials and Geometries}
    
\scriptsize
\begin{columns}
\begin{column}{0.26\textwidth}
    \centering
    \includegraphics[width=\columnwidth]{img/materials.png}
            
\end{column}
\begin{column}{0.65\textwidth}
A square cell region is used to surround each pin, with \textbf{transparent boundary conditions}.
    \centering
    \includegraphics[width=\columnwidth]{img/single_assembly_geom.png}

\end{column}
\end{columns}

\end{frame}

\begin{frame}{Cells and Universes}
    

    \begin{columns}
\begin{column}{0.6\textwidth}

    \centering
    \includegraphics[width=\columnwidth]{img/cell_single_assembly.png}
            
\end{column}
\begin{column}{0.4\textwidth}
    Regions of uniform composition are assigned to  each \textbf{cell}. 
    These regions can be defined through the use of Boolean operators.
    The possibility to use \textbf{universes} enables to model any identical repeated structures and then fill them with different elements.
\end{column}
\end{columns}
    

\end{frame}

%\begin{frame}{Cells and Universes}
%
%    
%\begin{columns}
%\begin{column}{0.6\textwidth}
%\centering
%
%   \includegraphics[width=\columnwidth]{img/guide_tubes.png}
%
%\end{column}
%\begin{column}{0.3\textwidth}
%\centering
%
%\end{column}
%\end{columns}
%
%
%\end{frame}

\begin{frame}{Fuel Assembly}

    
Finally, to obtain the Fuel Bundle the \textbf{lattice} has been filled with all the universes defined.
\vfill

\begin{columns}
\begin{column}{0.7\textwidth}


\centering
\includegraphics[width=\columnwidth]{img/bundle_universe.png}
            
\end{column}
\end{columns}


\end{frame}

\begin{frame}{Plots}

    \vspace{-30pt}
\begin{center}
        \includegraphics[angle=-90,width=\linewidth]{img/full_ass.png}
    \end{center}

    % Bottom row: two side-by-side images
    \begin{center}
        \begin{minipage}{0.30\linewidth}
            \includegraphics[angle=-90,width=\linewidth]{img/assembly_lower.png}
        \end{minipage}
        \hfill
        \begin{minipage}{0.56\linewidth}
            \includegraphics[angle=-90,width=\linewidth]{img/assembly_upper.png}
        \end{minipage}
    \end{center}
%\end{columns}     
\end{frame}

\begin{frame}{Plots}
    
    \begin{columns}
    \begin{column}{0.5\textwidth}
        \centering
        \includegraphics[width=\columnwidth]{img/assembly_xy.png}
    \end{column}
    \begin{column}{0.5\textwidth}
        \centering
        \includegraphics[width=0.9\columnwidth]{img/gt.png}
    \end{column}

\end{columns}
\end{frame}


\begin{frame}{Tallies}
    Any of the \textbf{tally} $X$ could be defined as:
    \begin{equation*}
        X = \underbrace{\int d\mathbf{r} \int d \boldsymbol{\Omega} \int
    dE}_{\text{filters}} \underbrace{f(\mathbf{r}, \boldsymbol{\Omega},
    E)}_{\text{scores}} \psi (\mathbf{r}, \boldsymbol{\Omega}, E)
    \end{equation*}

A \textbf{filter} identify which regions of phase space should score to a given tally as well as the \textbf{scoring function} $f$.

The \textbf{estimator} selected in this case is the \textbf{Track-lenght} estimator. 
\begin{equation*}
    V \phi = \frac{1}{W} \sum_{i \in T} w_i \ell_i
\end{equation*}

\end{frame}

\begin{frame}{Tallies}
\only<2>{
\vspace{-15pt}
\begin{columns}
    \begin{column}{\textwidth}
        \centering
        \includegraphics[width=0.8\columnwidth]{img/tallies.png}
    \end{column}
    
\end{columns}
}
\only<1>{
\textbf{Filters:}
\begin{itemize}
    \item \textbf{energy filter}, two energy intervals are used $[0,0.68]eV$ and $[0.68,2\times10^6] eV$
    \item \textbf{mesh filter}, the number of mesh cells is equal to the number of fuel pins in the assembly.
                Along the axial direction is used a single bin.
\end{itemize}
\textbf{Scores:}
\begin{itemize}
    \item \textbf{flux}, rappresents the total flux, measured in \textit{particles-cm per source particles};
    \item \textbf{kappa-fission}, the recoverable energy production rate due to fission:
                sum of fission product kinetic energy, kinetic energy of
                prompt and delayed neutrons, prompt and delayed $\gamma$-rays. Units are $eV$ per source particle.
    
\end{itemize}
}

\end{frame}

\section{Sensitivity Analysis}
\begin{frame}{Simulation Settings}
Specify \textbf{execution settings}:
\begin{itemize}
    \item \textbf{run mode}, has been choosen the \textbf{eigenvalue} mode. 
    The transport equation becomes an eigenvalue equation: since then the source will depend on the flux of neutrons itself. 
    \item \textbf{source distribution};
    \item \textbf{batches}, rappresents the number of groups in which the source particle histories are broken up. 
                        By default, a batch will consist of only a single fission generation. \textbf{Inactive},
                          refers to the number of initial batches during which the code does not score tallies.
    \item number of \textbf{particles}, will determine the statistical uncertainty. A higher number of particles will reduce the uncertainty.
\end{itemize}
\end{frame}

\begin{frame}{Sensitivity analysis}
\begin{columns}
        \begin{column}{0.5\textwidth}
        \centering
          \includegraphics[width=\columnwidth]{img/opt_batches_function.png}   
        \end{column} 
        \begin{column}{0.5\textwidth}
        \vspace{-20pt}
        \scriptsize
        \begin{itemize}
            \item \textbf{N}, number of cycles, $N=5$;
            \item \textbf{batch range}, interval (min batches, max batches) to test, $(50,100)$;
            \item \textbf{inactive}, the percentage of inactive batches $(20\%/40\%)$;
            \item \textbf{source}, a source region equal to the whole fuel bundle, 
                    selecting only the fissionable material;
            \item \textbf{particles}, number of particles per batch.
        \end{itemize}
        \centering
        \includegraphics[width=\columnwidth]{img/optimized_batches (2).png}
    \end{column}
\end{columns}
\end{frame}

\begin{frame}{Sensitivity analysis}
\only<1>{
    \begin{columns}

    \begin{column}{0.5\textwidth}
    \begin{figure}
        \centering
        \includegraphics[width=\columnwidth]{img/sensitivity_20.png}
        \caption{ $k_{eff}= 1.4060\pm 0.0004$ that means a relative error $\gamma _k=3.1\times 10^{-4}$}
    \end{figure}
    \end{column} 
    
    \begin{column}{0.5\textwidth}
        \begin{figure}
            \centering
        \includegraphics[width=\columnwidth]{img/sensitivity_40.png}
        \caption{ $k_{eff}=1.4052\pm0.0004$ that means a relative error $\gamma _k=2.8 \times 10^{-4}$}    
        \end{figure}
        
    \end{column}
\end{columns}
}
\only<2>{
    From previous graphs:
    \begin{itemize}
        \item The value of $\boldsymbol{\gamma_k}$ decrases as the number of batches increases;
        \item A higher $\boldsymbol{\gamma_k}$ may indicate an influence from source
        instabilities. However, the \textbf{relative error} remains more stable with a lower number
        of inactive batches.
        \item $10^5$ particles were used because of limited computational capabilities. 
    \end{itemize} 
    \begin{columns}
        
        \begin{column}{0.5\textwidth}
            
            \centering
            \includegraphics[width=\columnwidth]{img/simulation_set.png}
            
        \end{column}
        
    \end{columns}

}
\end{frame}

\section{Running the Simulation}
\begin{frame}{Fluxes comparison}
 \begin{columns}
     
    \begin{column}{0.62\textwidth}
        
        \includegraphics[width=\columnwidth]{img/fluxes_normal_enr.png}
        
    \end{column}
    \begin{column}{0.4\textwidth}
        Upper map Zircaloy bundle; lower map Fecral bundle. 
        
\begin{table}
    \centering
    \renewcommand{\arraystretch}{1.2}
    \resizebox{\linewidth}{!}{  % Resize table to fit frame
    \begin{tabular}{lcc}
        \toprule
        \textbf{Assembly} & \textbf{Thermal flux 
        } [cm] & \textbf{Fast flux }[cm] \\
        \midrule
        
        Zircaloy & 0.012-0.018 & 0.143-0.147\\
        FeCrAl &0.011-0.017&0.142-0.146\\
        \bottomrule

    \end{tabular}
    }
\end{table}
        To better visualize the variation between the two bundles, 
        the ratio of the fluxes for both energy bins has been plotted.
    \end{column}
    
\end{columns}
\end{frame}

\begin{frame}{Fluxes comparison}
   
    \begin{columns}

        \begin{column}{0.5\textwidth}
        
            \centering
            \includegraphics[width=\columnwidth]{img/normalized_thermal_flux.png}
        
        \end{column}
        \begin{column}{0.5\textwidth}
        
            \centering
            \includegraphics[width=\columnwidth]{img/normalized_fast_flux.png}
        
        \end{column}
    
    \end{columns}
\begin{columns}
    \begin{column}{0.3\textwidth}
        
        \begin{table}
            \centering
            \renewcommand{\arraystretch}{1.2}
            \resizebox{\columnwidth}{!}{ 
                \begin{tabular}{cc}
                    \toprule
                    \textbf{Thermal flux 
                    }  & \textbf{Fast flux } \\
                    \midrule
                     0.910-0.946 & 0.987-1.000\\
                    \bottomrule
                    
                \end{tabular}
                }
            \end{table}
    \end{column}
\end{columns}
    
\end{frame}

\begin{frame}{PPF evaluation}
\only<1>{
    \begin{columns}
    
    \begin{column}{0.6\textwidth}
        
            The PPF is defined as:
            \begin{equation*}
                PPF=\frac{P_{max}}{P_{avg}}.
            \end{equation*}
            Since the power produced per pin is proportional to the flux, 
            the PPF should be independent of it, implying that FeCrAl and Zircaloy 
            PPFs should be similar.
            The tally score used was the \textbf{kappa-fission}. 
        
        \end{column}
           
    \begin{column}{0.4\textwidth}

            \centering
            \includegraphics[width=\columnwidth]{img/ppf_eval.png}

    \end{column}
    \end{columns}
}
\only<2>{
    \vspace{-10pt}
   
    \begin{columns}

        \begin{column}{0.5\textwidth}
        
                \centering
            \includegraphics[width=\columnwidth]{img/pdm_zircaloy_normal.png}
        
        \end{column}
        \begin{column}{0.5\textwidth}
        
            \centering
            \includegraphics[width=\columnwidth]{img/pdm_fecral_normal.png}
        
        \end{column}
    
    \end{columns}
    \begin{columns}
        \begin{column}{0.7\textwidth}
            \begin{table}
            \centering
            \renewcommand{\arraystretch}{1.2}
            \resizebox{0.5\columnwidth}{!}{ 
                \begin{tabular}{cc}
                    \toprule
                    \textbf{Zircaloy range
                    }  & \textbf{FeCrAl range } \\
                    \midrule
                     0.888-1.085 & 0.884-1.090\\
                    \bottomrule
                    
                \end{tabular}
                }
            \end{table}
        \end{column}
    \end{columns}
}
\end{frame}

\begin{frame}{Power distribution comparison}
    
    
\begin{columns}     
    \begin{column}{0.6\textwidth}
        
        \centering
        \includegraphics[width=\columnwidth]{img/norm_nromal_enr.png}
    \end{column}
    \begin{column}{0.4\textwidth}
        
        The power distribution of Fecral bundle 
        has been normalized respect the power distribution of Zircaloy.
       
       
        \textbf{Range}: $0.914-0.954$ 


        The values in the power map are similar to the thermal flux map. 
        There is a slightly mismatch due to the \textbf{kappa-fission} score. 
    
    \end{column}
\end{columns}
\end{frame}

\begin{frame}{Enhancement of the assembly: strategies}
\begin{columns}
    \begin{column}{0.5\textwidth}
        
        Another enrichment distribution has been adopted for the fuel assembly, following come constraints:
        \begin{itemize}
            \item respect the simmetry of the problem;
            \item mantain the \textbf{average enrichment} at $5\%$;
            \item try to \textbf{flatten the flux and reduce PPF};
        \end{itemize}
        The attempt with three different enrichments $(6.20\%, 5.45\%, 4.6\%)$ is shown in the figure.
    \end{column}
    
    \begin{column}{0.5\textwidth}
        \vspace{-5pt}
        \centering
        \includegraphics[width=\columnwidth]{img/three_enr.png}
        
    \end{column}
\end{columns}
\end{frame}

\begin{frame}{Enhancement of the assembly: Fluxes}
\begin{columns}     
    \begin{column}{0.6\textwidth}
        
        \centering
        \includegraphics[width=\columnwidth]{img/new_enr.png}
        
    \end{column}
    \begin{column}{0.4\textwidth}
        Upper map Zircaloy bundle; lower map Fecral bundle. 
    \begin{table}
    \centering
    \renewcommand{\arraystretch}{1.2}
    \resizebox{\linewidth}{!}{  % Resize table to fit frame
    \begin{tabular}{lcc}
        \toprule
        \textbf{Assembly} & \textbf{Thermal flux 
        } [cm] & \textbf{Fast flux }[cm] \\
        \midrule
        
        Zircaloy & 0.011-0.019 & 0.143-0.148\\
        FeCrAl &0.010-0.018&0.142-0.147\\
        \bottomrule

    \end{tabular}
    }
\end{table}
    \end{column}
\end{columns}
\end{frame}

\begin{frame}{Enhancement of the assembly: Fluxes}
    
    \begin{columns}

        \begin{column}{0.5\textwidth}
        
                \centering
                \includegraphics[width=\columnwidth]{img/thermal_flux_right_enr.png}
                
            \end{column}
            \begin{column}{0.5\textwidth}
                
                \centering
                \includegraphics[width=\columnwidth]{img/fast_flux_right_enr.png}
        
        \end{column}
    
    \end{columns}
    \begin{columns}
    \begin{column}{0.3\textwidth}
        
        \begin{table}
            \centering
            \renewcommand{\arraystretch}{1.2}
            \resizebox{\columnwidth}{!}{ 
                \begin{tabular}{cc}
                    \toprule
                    \textbf{Thermal flux 
                    }  & \textbf{Fast flux } \\
                    \midrule
                     0.907-0.947 & 0.989-1.001\\
                    \bottomrule
                    
                \end{tabular}
                }
            \end{table}
    \end{column}
\end{columns}
    
\end{frame}

\begin{frame}{Enhancement of the assembly: Power map}
\begin{columns}     
    \begin{column}{0.5\textwidth}
        
        \centering
        \includegraphics[width=\columnwidth]{img/pdm_zircaloy_right_enr.png}
        
    \end{column}
    \only<1>{\begin{column}{0.5\textwidth}
        
        \centering
        \includegraphics[width=\columnwidth]{img/pdm_fecral_right_enr.png}
        
    \end{column}}
    \only<2>{\begin{column}{0.5\textwidth}
        \centering
        \includegraphics[width=\columnwidth]{img/pdm_zircaloy_normal.png}
        
    \end{column}}
\end{columns}
\only<1>{\begin{columns}
        \begin{column}{0.7\textwidth}
            \begin{table}
            \centering
            \renewcommand{\arraystretch}{1.2}
            \resizebox{0.5\columnwidth}{!}{ 
                \begin{tabular}{cc}
                    \toprule
                    \textbf{Zircaloy range
                    }  & \textbf{FeCrAl range } \\
                    \midrule
                     0.925-1.052 & 0.925-1.052\\
                    \bottomrule
                    
                \end{tabular}
                }
            \end{table}
        \end{column}
    \end{columns}
}
\end{frame}

\begin{frame}{Enhancement of the assembly: Normalized map}
\begin{columns}     
    \begin{column}{0.7\textwidth}
        
        \centering
        \includegraphics[width=\columnwidth]{img/normalized_pdm.png}
    \end{column}
    \begin{column}{0.4\textwidth}
        \textbf{Range}: $0.916-0.954$
    \end{column}
\end{columns}
\end{frame}

\section{Results \& Discussion}
\begin{frame}{Further Improvements}
    \begin{itemize}
        \item The geometries adopted were strongly simplified, a \textbf{higher degree of 
                detail} could be achieved;
        \item a \textbf{higher number of particles} should be implemented in order to reduce
                fluctuation effects. These effects were mainly visible plotting the flux maps;

        \item increasing the number of particle means also to increase the computational cost. 
                Then, this solution should be accomplished exploiting the symmetries of the problem.
    \end{itemize}
    
\end{frame}
\begin{frame}{Conclusion}
   Primarily, due to its \textbf{absorption microscopic cross section}, FeCrAl exhibits a lower $k_{eff}$ and
     reduced power production per fuel pin.
   \begin{itemize} 
       \item[\ding{227}] However, evaluating the \textbf{PPF for both bundles}, 
                           a similar behavior has been obtained, implying \textbf{a similar consumption of 
                           fissile material}. \textbf{This is true also using 
                           a different enrichment pattern}, but with a uniform consumption of fuel.
       \item[\ding{227}] To overcome the problem related to the microscopic CS is possible to
                    \textbf{reduce the thickness of the clad}, respecting the mechanical limit of the material.
       \item[\ding{227}] Finally, the main objective of ATF should be the capability to withstand against
                   a accident condition, thanks to some solutions proposed also here, the 
                   FeCrAl alloys remains promising candidates.
   \end{itemize}
\end{frame}

\begin{frame}{Bibliography}
    \nocite{*}
    \printbibliography
\end{frame}

\end{document}