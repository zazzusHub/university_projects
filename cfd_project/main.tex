\documentclass[serif]{beamer}
\usepackage[T1]{fontenc}
\usepackage[sfdefault]{FiraSans}
\usepackage[english]{babel}
\usepackage[utf8]{inputenc}
\usepackage[separate-uncertainty=true]{siunitx}
\usepackage{amsmath}
\usepackage{amssymb}
\usepackage{amsfonts}
\usepackage{mathrsfs}
\usepackage{mathtools}
\usepackage{commath}
\usepackage{braket}
\usepackage{graphicx}
\usepackage[euler-digits]{eulervm}
\usepackage{pifont} 
\usepackage{comment}
\usepackage{booktabs}
\usepackage{tikz}
\usepackage[backend=biber]{biblatex} % o style=authoryear
\addbibresource{biblio.bib}
\usetikzlibrary{shapes.geometric, arrows.meta, positioning, calc}


% Configurazione Tema
\usetheme{Madrid}
\useoutertheme[subsection=false]{miniframes}

% --- DEFINIZIONE COLORI STAR-CCM+ AGGIORNATI ---
\definecolor{StarCCMBlue}{RGB}{0, 60, 75}      % Blu scuro istituzionale
\definecolor{StarCCMCyan}{RGB}{0, 176, 202}     % Ciano acceso
\definecolor{StarCCMGray}{RGB}{200, 210, 215}   % Grigio più scuro per titoli/contrasto

% --- LOGICA LOGO (Nascondi in TitlePage) ---
\logo{%
    \ifnum\insertframenumber>1
        \includegraphics[height=1 cm]{sccm.png}%
    \fi
}

% --- APPLICAZIONE COLORI ---

% 1. Titoli delle slide (Nero su Grigio StarCCM)
\setbeamercolor{frametitle}{fg=white, bg=StarCCMBlue}

% 2. Fascia superiore (Sezioni/Miniframes)
\setbeamercolor{section in head/foot}{fg=black, bg=StarCCMGray}

% 3. Fascia inferiore (Footer in ordine cromatico)
\setbeamercolor{author in head/foot}{bg=StarCCMBlue, fg=white}   % Parte sinistra (Autore)
\setbeamercolor{title in head/foot}{bg=StarCCMCyan, fg=white}    % Parte centrale (Titolo)
\setbeamercolor{date in head/foot}{bg=StarCCMGray, fg=black}    % Parte destra (Data)

% 4. Struttura generale
\setbeamercolor{structure}{fg=StarCCMCyan}
\setbeamercolor{title}{fg=white, bg=StarCCMBlue}
\setbeamertemplate{frametitle}[default][center]

% --- INFORMAZIONI PRESENTAZIONE ---
\title[Moody's Diagram]{Distributed Pressure Drops: an Axisymmetric Approach}

\titlegraphic{
    \begin{columns}
        \begin{column}{0.5\textwidth}
            \centering
            \includegraphics[scale=0.17]{cherub.png}
        \end{column}
        \begin{column}{0.5\textwidth}
            \centering
            \includegraphics[scale=0.08]{sccm.png}
        \end{column} 
    \end{columns}
}

\author[F. Zazzu]{Francesco Zazzu}
\date[2 february 2026]{A.Y. 2024/2025}
\institute[]{MSc in Nuclear Engineering \\ Università di Pisa}

\begin{document}

% Prima slide (senza logo)
\begin{frame}
    \titlepage
\end{frame}

\section{Introduction}
 
\begin{frame}{Introduction}
    \textbf{Distributed Pressure Drops} are estimated using \textbf{Experimental Relationships} to 
    define \textbf{Friction Factors} dependent on flow, fluid parameters and the specific application.  
    
    The aims of this work are:
    \begin{itemize}
        \item [\ding{227}] Evaluate through a \textbf{CFD code} the friction factors in a pipe of circular cross section. 
        \item [\ding{227}] Identify the factors affecting the CFD simulation result. 
        \item [\ding{227}] Compare the results with the experimental relationships.
    \end{itemize} 

   The simulations were performed using StarCCM+.
\end{frame}

\section{CFD computation: laminar regime and smooth pipe}
\begin{frame}[c]
    \Large
    \centering
    \textit{CFD computation: Laminar regime and Turbulent smooth pipe}
\end{frame}

\begin{frame}{Geometry}


    A circular pipe was modeled using a \textbf{2D axisymmetric approach}.
    
    Length $L=2.25 m$ and diameter $D=3.75$ cm $\implies L/D=60$. 
\begin{columns}
    \vspace{0.3 cm}
    \begin{column}{0.75\textwidth}
        \centering
        \includegraphics[width=\columnwidth]{geometry2.png}
    \end{column}   
\end{columns}

\end{frame}

 \begin{frame}{Boundary Conditions}
    The fluid used is water with constant density $\rho =997.561 \unit{\kilo\gram\per\cubic\metre}$ 
    and constant dynamic viscosity $\mu =8.8871\times 10^{-4} \unit{\pascal\second}$. 
    The \textbf{Reynolds} numbers selected for the laminar and turbulent region were: 
    \footnotesize 
    \begin{align*}
        Re_{laminar}   &= (1000, 1500, 2000) \\
        Re_{turbulent} &= (4000, 16000, 20000, 45000, \\
                       &\quad 63000, 250000, 1000000)
    \end{align*}
    \normalsize

    The \textbf{Inlet Type} was imposed as \textbf{Velocity Inlet}, while the 
    \textbf{Outlet Type} was set as a \textbf{Pressure Outlet}.
    
\end{frame}

\begin{frame}{Mesh}

    The mesh selected for the simulation is an \textbf{Automated Axisymmetric Two-Dimensional} mesh.
    In particular, a \textbf{Quadrilateral Mesher} is selected for the center of the pipe and a \textbf{Prism Layer 
    Mesher} for the region near the wall. 
\begin{columns}
    \begin{column}{0.4\textwidth}
        \centering
        \includegraphics[width=\columnwidth]{mesh.png}
    \end{column}
    \begin{column}{0.6\textwidth}
        Properties:
        \begin{itemize}
            \item \textbf{Base size}: $ 2 mm$;
            \item Prism layer \textbf{Total Thickness}: $15 mm$;
            \item \textbf{Number of prism layers}: $10$;
            \item Prism layer  \textbf{Stretching}: $1.2$.
        \end{itemize}
    \end{column}
\end{columns}

\end{frame}

\begin{frame}{On the selection of the Mesh}
    The chosen properties depend on:
\begin{itemize}
            \item \textbf{Reynolds number and velocity inlet}, higher velocities enhance the turbulence;
            \item \textbf{Wall Y+} values, it is common to have $Y^+=30$ in the center of the first cell on the wall;
            \item The \textbf{Physics Models} used in the simulation;
            \item \textbf{Wall roughness}, a thickness of the first prismatic cell comparable with the roughness height of the pipe leads to a 
                   strong variation in the results.
        \end{itemize}
\end{frame}

\begin{frame}{Wall Y+ estimates}
    \only<1>{
        At which distance $y$ from the wall does $Y^+=30$ appear?
        An initial estimate of $y$ comes from the relation: 
        \begin{equation}
            f=8\frac{\tau _w}{\rho w^2} \implies y=\frac{Y^+ D}{Re \sqrt{f/8}} 
        \end{equation}
        $f$ was evaluated using the known relationships. Meanwhile, the first cell height $h_1$ is calculated as: 
        \begin{equation}
            h_1=H_{Tot}\frac{r-1}{r^n-1}
        \end{equation}
        Where $r$ is the prism layer stretching, $n$ the number of prism layers and  $H_{tot}$ the total thickness.
        These two values are then compared. For the chosen properties $h_1=0.58 $mm.
    }
   \only<2>{
    \begin{table}[htbp]
    \centering
    \caption{Computed distance at which $Y^+=30$ is expected and effective $Y^+$ value for $h_1 = 0.58 $mm.}
    %\label{tab:yplus_analysis}
    %\vspace{0.2cm} % Piccola spaziatura tra caption e tabella
    
    \begin{tabular}{
        S[table-format=7.0] % Colonna 1: Re (fino a 7 cifre intere)
        S[table-format=3.2] % Colonna 2: Delta y (3 interi, 2 decimali)
        c                   % Colonna 3: y+ (centrata perché contiene simboli come < o approx)
    }
        \toprule
        {Reynolds} & {$ y (Y^+=30)$ [\unit{mm}]} & {$Y^+$ for $y=h_1/2$} \\
        \midrule
        4000    & 4 & $ 2.2$ \\
        16000   & 1.2  & $ 7.5$ \\
        20000   & 0.98  & $ 9$ \\
        45000   & 0.48   & $ 18$ \\
        63000   & 0.36   & $ 25$ \\
        250000  & 0.1  & $ 94$ \\
        1000000 & 0.03   & $ 293$ \\
        \bottomrule
    \end{tabular}
\end{table}
   }
   \only<3>{
        Which is the most suitable model for the \textbf{wall treatment}?

        Different treatments could be adopted in StarCCM+:
        \begin{itemize}
            \item \textbf{High Y+}, assumes $Y^+>30$, permits the use of a coarse mesh, less time consuming;
            \item \textbf{Low Y+}, assumes $Y^+<5$, a fine layered mesh should be used, computationally demanding;
            \item \textbf{All Y+}, this treatment could adapt the Wall function to the type of mesh adopted,
                                    for intermediate meshes and works fine for $5<Y^+<30$;
            \item \textbf{Two Layer All Y+}, same as All Y+, the turbulent dissipation rate is imposed at the center of the first cell.
        \end{itemize}
        The \textbf{Two Layer All Y+} treatment was selected for all simulated cases.
   }
\end{frame}

\begin{frame}{The Physics Model}
\only<1>{
The \textbf{Two layer All Y+} approach overcomes the $Y^+=30$ problem. The adoption of this treatment imposes the use of Realizable $k-\epsilon$ model.
    \begin{columns}
\begin{column}{0.5\textwidth}

    \centering
    \includegraphics[width=0.75\columnwidth]{laminar_phys.png}
            
\end{column}
\begin{column}{0.5\textwidth}
    \centering
    \includegraphics[width=0.65\columnwidth]{turbulent_phys.png}
\end{column}
\end{columns}
}
\only<2>{
    Wall $Y^+$ computed with the \textbf{Two Layer All Y+ approach}.
    \begin{columns}
    \begin{column}{0.5\textwidth}
        \centering
        \includegraphics[width=\columnwidth]{wally_4000.png}
    \end{column}
    \begin{column}{0.5\textwidth}
        \centering
        \includegraphics[width=\columnwidth]{wallplus_63mil.png}
    \end{column}
\end{columns}
}

\end{frame}

\begin{frame}{Derived parts and $\Delta P$ determination}
\only<1>{
    The region of the pipe with a \textbf{fully developed flow} is the one selected for the estimation of the pressure drop $\Delta P$.
The velocity profile was observed. Was the pipe long enough?
\begin{columns}
    
\begin{column}{0.5\textwidth}

\centering
\includegraphics[width=\columnwidth]{velocity_63k.png}
            
\end{column}
\begin{column}{0.5\textwidth}

\centering
\includegraphics[width=\columnwidth]{velocity_63k_long.png}
            
\end{column}
\end{columns}
}
\only<2>{
\begin{columns}
\begin{column}{0.9\textwidth}

\centering
\includegraphics[width=\columnwidth]{long.png}
            
\end{column}
\end{columns}

}
\only<3>{
    Creating some \textbf{derived parts}, i.e. plane sections, it was possible to run a report and estimate the $\Delta P$.
    \begin{table}[htbp]
        \centering
        % CORREZIONE QUI SOTTO: Aggiunto {cc} per definire 2 colonne centrate
        \begin{tabular}{cc} 
            \toprule
            $\Delta P\quad$ L/D=60 & $\Delta P\quad$ L/D=120 \\
            \midrule
            $0.107 \unit{\kilo\pascal}$ & $0.108 \unit{\kilo\pascal}$ \\
            \bottomrule
        \end{tabular}
    \end{table}
    The local pressure has been evaluated at $1.8$ m and $2$ m. 
    $\Delta L=0.2$ m was used for all cases.
    
    \vspace{0.5 cm}
    
    \begin{columns}
        \begin{column}{0.7\textwidth}
            \centering
            \includegraphics[width=\columnwidth]{derived_part.png}     
        \end{column}
    \end{columns}
}


\end{frame}

\begin{frame}{Friction factors evaluation}

    To compute the friction factors coming from  the simulation,$f_{CFD}$, 
    the \textbf{Darcy-Weisbach} relation was used: 
    \begin{equation*}
        f_{CFD}=2\frac{\Delta P D }{\rho v^2 \Delta L}
    \end{equation*}
    while, as a reference, the \textit{Poiseuille, Blasius and McAdams}
    relationships were adopted. 
    \begin{align*}
    f_{\text{Poiseuille}} &= 64/\mathrm{Re}          && \text{Laminar Regime}\\
    f_{\text{Blasius}}    &= 0.316\mathrm{Re}^{-0.25} && \text{Smooth, } \mathrm{Re} < 30000 \\
    f_{\text{McAdams}}    &= 0.184\mathrm{Re}^{-0.2}  && \text{Smooth, } 30000 < \mathrm{Re} < 10^6
\end{align*}

\end{frame}

\begin{frame}{Results}

\only<1>{
    \vspace{-0.8cm}
    \begin{columns}
    \begin{column}{\textwidth}
        \centering
        \includegraphics[width=\columnwidth]{laminar_smooth_relationships.png}
    \end{column}
\end{columns}
 }
 \only<2>{\begin{columns}
    
    \begin{column}{\textwidth}
        \centering
        \includegraphics[width=0.8\columnwidth]{smooth_laminar_errrel.png}
    \end{column}
\end{columns}}
\only<3>{
    Some comments on the results:
    \begin{itemize}
        \item The discrepancy between the simulation values and the relationship predictions grows as the 
                Reynolds number approaches the \textbf{Transition Region}.
        \item For $20000<Re<250000$ the simulations underestimate the friction factors values.
        \item Note the high relative error for $Re=4000$. 
    \end{itemize}
}
\end{frame}

\section{CFD computation: rough pipe cases}
\begin{frame}[c]
    \Large
    \centering
    \textit{CFD computation: Turbulent regime Rough pipe}
\end{frame}

\begin{frame}{Roughness and pre-processing}
    \textbf{Selected Relative Roughness} values: 
    \begin{align*}
    \epsilon /D &= (0.0001, 0.001, 0.002, 0.005, 0.01) \\
    \epsilon    &= (3.75\times 10^{-6}, 3.75 \times 10^{-5}, 7.5 \times 10^{-5}, \\
                &\quad 1.875 \times 10^{-4}, 3.75 \times 10^{-4}) \, \text{m}
\end{align*}
    It is evident that the last two \textbf{roughness height} values are comparable to $h_1$.
    How does the simulation behave with the same physics and mesh? 
\end{frame}

\begin{frame}{Reference Relationships}
    To evaluate $f_{CFD}$ the Darcy-Weisbach relation is used.
    For the analytical comparison, the following relationships were adopted:

    % Riduciamo il font e lo spazio verticale
    \footnotesize 
    \vspace{0.2cm}
    
    % --- BLOCCO 1: CHURCHILL ---
    \textbf{1. Churchill} ($Re = 4000$):
    \vspace{-0.5em}
    \begin{align*}
        f &= 8 \left[ \left( \frac{8}{Re} \right)^{12} + \frac{1}{(A+B)^{1.5}} \right]^{\frac{1}{12}} \\
        \text{where:} \quad 
        A &= \left[ 2.457 \ln \left\{ \frac{1}{(7/Re)^{0.9} + 0.27(\varepsilon/D)} \right\} \right]^{16} \\
        B &= \left[ \frac{37530}{Re} \right]^{16}
    \end{align*}

    \vspace{-0.2cm} % Riduciamo spazio tra i due blocchi

    % --- BLOCCO 2: SWAMEE-JAIN ---
    \textbf{2. Swamee-Jain} ($Re > 4000$):
    \vspace{-0.5em}
    \begin{equation*}
        f = \frac{0.25}{\left[ \log_{10}\left( \frac{1}{3.7(D/\varepsilon)} + \frac{5.74}{Re^{0.9}} \right) \right]^2}
        \quad 
        % Condizioni scritte con cases (più pulito)
        \text{valid for }
        \begin{cases}
            5000 < Re < 10^8 \\ 
            10^{-6} < \varepsilon/D < 10^{-2}
        \end{cases}
    \end{equation*}

\end{frame}

\begin{frame}{Results}
    \only<1>{
    \begin{columns}
    
    \begin{column}{0.95\textwidth}
        \centering
        \includegraphics[width=\columnwidth]{turbulent_part_brutto.png}
    \end{column}
\end{columns}
 }
 \only<2>{\begin{columns}
    
    \begin{column}{\textwidth}
        \centering
        \includegraphics[width=0.8\columnwidth]{relative_error_mesh1.png}
    \end{column}
\end{columns}}
\only<3>{Some comments on the results:
\begin{itemize}
    \item In general, the simulations understimated the friction factors compared to the relationships.
    \item As anticipated, the \textbf{largest discrepancies} between references and simulations appear for the points with \textbf{highest 
            relative roughness} values.
    \item The formation of a dip region for $16000<Re<63000$ is evident. 
\end{itemize}
}
\end{frame}

\begin{frame}{New mesh settings}
The similar values of roughness height and first cell height suggest that the \textbf{mesh is inadequate} to resolve the turbulence. 
Reducing the number of prism layers for the two higher roughness cases:
\begin{columns}
    \begin{column}{0.4\textwidth}
        \centering
        \includegraphics[width=\columnwidth]{mesh2.png}
    \end{column}
    \begin{column}{0.6\textwidth}
        Properties:
        \begin{itemize}
            \item \textbf{Base size}: $ 2 mm$;
            \item \textbf{Prism layer Total Thickness}: $15 mm$;
            \item \textbf{Number of prism layer}: $5$;
            \item \textbf{prism layer Stretching}: $1.2$;
            \item first cell height $h_1\approx 2 $mm.
        \end{itemize}
    \end{column}
\end{columns}
\end{frame}

\begin{frame}{New results}
\only<1>{
    \begin{columns}
    \vspace{-0.8 cm}
    \begin{column}{\textwidth}
        \centering
        \includegraphics[width=0.95\columnwidth]{turbulent_part_bello.png}
    \end{column}
\end{columns}
 }
 \only<2>{\begin{columns}
    
    \begin{column}{\textwidth}
        \centering
        \includegraphics[width=0.8\columnwidth]{relative_error_rough.png}
    \end{column}
\end{columns}}
\only<3>{Some comments on the results:
\begin{itemize}
    \item With the new mesh the discrepancies are reduced.
    \item Again, the simulations underestimated the friction factor.
    \item The phenomenon in the region $16000<Re<63000$ still remains. This phenomenon is related to how the code deals with the roughness.
          In particular, StarCCM+ reproduces a similar trend to Nikuradse's experiment.
\end{itemize}
}
\end{frame}

\begin{frame}{On Nikuradse}
    
    \begin{columns}
    %\vspace{-0.8 cm}
    \begin{column}{0.6\textwidth}
        \centering
        \includegraphics[width=\columnwidth]{nikuradse.png}
    \end{column}
    \begin{column}{0.4\textwidth}
       Nikuradse created artificially roughened pipes to control the geometry of the surface irregularities.
       He achieved this by gluing uniform sand grains of known sizes onto the inner walls of the pipes. 
       We can recognise the same behaviour in the graph on the left.  
    \end{column}
\end{columns}
\end{frame}

\begin{frame}{What about changing the RANS model?}
\only<1>{
    Moving from a $k-\epsilon$ to a $k-\omega$ model, an 
    \textbf{All Y+} model was adopted. The mesh remains the same.
    %-0.2 cm\vspace{}
    \begin{columns}
        \begin{column}{\textwidth}
        \centering
        \includegraphics[width=0.72\columnwidth]{omega_vs_epsilon.png}
    \end{column}
\end{columns}
 }
 \only<2>{\begin{columns}
    
    \begin{column}{\textwidth}
        \centering
        \includegraphics[width=0.8\columnwidth]{omega_vs_epsilon_err.png}
    \end{column}
\end{columns}}
\end{frame}

\begin{frame}{Total result}
    \vspace{-0.8cm}
    \begin{columns}
    \begin{column}{\textwidth}
        \centering
        \includegraphics[width=\columnwidth]{tot_finale.png}
    \end{column}
\end{columns}
\end{frame}

\section{Results \& Discussion}
\begin{frame}{Conclusion}
   This work explored the possibility of computing the Friction-Factor through an axisymmetric approach.
\begin{itemize}
    \item[\ding{227}] \textbf{One of the factors affecting the results is the mesh choice}:
                     the choice of the \textbf{Wall Y+ treatment} requires for \textbf{low Reynolds} numbers a \textbf{coarser mesh}; for \textbf{high Reynolds} numbers
                        the \textbf{turbulent Boundary layer is thinner}, and a \textbf{finer mesh} is required; the \textbf{roughness} value requires a minimum height of the first cell.
    \item[\ding{227}] StarCCM+ reproduces the \textit{sand-grain} trends of Nikuradse, a different approach respect the relationships taken as references.
    \item[\ding{227}] Besides all the different aspects that are related to the discrepancies, the \textbf{RANS code and relationships are associated with an intrinsic error}, 
                        due to the complexity of fluid dynamics problems.  
    
\end{itemize}
\end{frame}

\begin{frame}{Bibliography}
    \nocite{*} % Cita tutte le voci nel .bib anche se non citate nel testo
    \printbibliography
\end{frame}



\end{document}
